\chapter{Gesti\'on de riesgos.}
\newpage
\section{An\'alisis de riesgos.}

Los riesgos que han podido ser identificados se detallan a continuaci\'on, indicandose a que tipo pertenecen:

\begin{itemize}
\item {\bf Riesgos T\'ecnicos}:
  \begin{itemize}
  \item[{\bf RT1.}] Errores de inicio y mantenci\'on de conexi\'on v\'ia Bluetooth entre aplicaci\'on y sistema rob\'otico.
  \item[{\bf RT2.}] Problemas de compatibilidad de hardware.
  \item[{\bf RT3.}] Inconsistencia entre robot f\'isico y mascota virtual.
  \item[{\bf RT4.}] P\'erdida de acceso al robot, ya sea por robo o falla t\'ecnica.
  \end{itemize}
\item {\bf Riesgos de Proyecto}:
  \begin{itemize}
  \item[{\bf RP1.}] Falta de experiencia de miembros del equipo en programaci\'on en ROBOTC y Android
  \item[{\bf RP2.}] P\'erdida de personal.
  \item[{\bf RP3.}] Problemas inesperados durante la implementaci\'on del proyecto.
  \item[{\bf RP4.}] Alto costo monetario de hardware o software necesario para la implementaci\'on.
  \end{itemize}
\item {\bf Riesgos de Negocio}:
  \begin{itemize}
  \item[{\bf RN1.}] Cese y desista por parte de LEGO.
  \item[{\bf RN2.}] Aplicaci\'on poco atractiva para p\'ublico objetivo.
  \end{itemize}
\end{itemize}

\section{Preparaci\'on para control de riesgos.}

Para poder hacer frente a los riesgos identificados, se detallan a continuaci\'on, indicando la prioridad, impacto, probabilidad, tipo de riesgo, contexto, plan de contingencia y de mitigaci\'on y la resoluci\'on de cada uno de ellos.

La simbolog\'ia asociada a cada detalle es:

\begin{itemize}
\item \emph{Prioridad e Impacto}: entre m\'as cercano a 1 es menor (escala de  1 a  5). 
\item \emph{Probabilidad}: entre m\'as cercano a 1 mayor. (escala de 0 a 1).
\end{itemize}

Se iniciar\'a indicando los riesgos de tipo T\'ecnicos, luego los de Proyecto y finalmente los de Negocio.

\newpage
\subsection{Riesgos T\'ecnicos}

%%
%%Riesgo RT1
%%
\begin{table}[htbp!]
  \centering
  \begin{tabular}{|p{4cm}p{1cm}|p{4cm}p{1cm}|p{4cm}p{1cm}|}\hline
    \multicolumn{2}{|m{5cm}}{\bf Riesgo}& \multicolumn{4}{m{11cm}|}{\justifying Errores de inicio y mantenci\'on de conexi\'on v\'ia Bluetooth entre aplicaci\'on y sistema rob\'otico.}\\\hline
    {\bf Prioridad}& 5& {\bf Impacto}& 5& {\bf Probabilidad}& 0.3\\\hline
    \multicolumn{2}{|m{5cm}}{\bf Tipo de Riesgo}& \multicolumn{4}{m{11cm}|}{\justifying T\'ecnico.}\\\hline
    \multicolumn{2}{|m{5cm}}{\bf Contexto}& \multicolumn{4}{m{11cm}|}{\justifying La interacci\'on de ambos sistemas se da exclusivamente por este medio, lo cual influye de manera considerable en la funcionalidad que se pueda obtener.
Se estima esencial la conexi\'on entre el robot y el smartphone ya que la correlaci\'on de ambas herramientas, as\'i como la fuente de su innovaci\'on, reside en su comunicaci\'on.}\\\hline
    \multicolumn{2}{|m{5cm}}{\bf Plan de Contingencia}& \multicolumn{4}{m{11cm}|}{\justifying El control de este riesgo es fundamental para el proyecto, ya sea lograr una conexi\'on exitosa, como mantenerla durante el tiempo de utilizaci\'on de la aplicaci\'on.\\
En caso de no poder cumplir uno de estos dos requerimientos se deber\'an tomar medidas para mitigar el problema, de no poder reemplazar la conexi\'on, se trabajara solo con uno de los dos aparatos (NXT o Smarthphone) lo que disminuye notablemente la innovaci\'on del proyecto.}\tabularnewline\hline
    \multicolumn{2}{|m{5cm}}{\bf Plan de Mitigaci\'on}& \multicolumn{4}{m{11cm}|}{\justifying De no lograrse una conexi\'on exitosa entre el NXT Intelligent Brick y el smartphone con Android:\begin{enumerate}\item se utilizar\'a un aparato m\'ovil con sistema operativo iOS permitiendo conservar la fuente de innovaci\'on tanto por parte de la interacci\'on entre el dispositivo y el robot, como la utilizaci\'on de sensores del smartphone para variadas funcionalidades.\item Se realizar\'a una conexi\'on v\'ia bluetooth con un computador.\item Se realizar\'a una conexi\'on v\'ia USB con el computador.\end{enumerate}}\\\hline
    \multicolumn{2}{|m{5cm}}{\bf Resoluci\'on}& \multicolumn{4}{m{11cm}|}{\justifying Al conectar exitosamente los dispositivos y mantener la conexi\'on se tendr\'a la base para llevar a cabo todo lo que demanda el proyecto.}\\\hline
  \end{tabular}
  \caption[~RT1]{}
  \label{table:RT1}
\end{table}


%%
%%Riesgo RT2
%%
\begin{table}[htbp!]
  \centering
  \begin{tabular}{|p{4cm}p{1cm}|p{4cm}p{1cm}|p{4cm}p{1cm}|}\hline
        \multicolumn{2}{|m{5cm}}{\bf Riesgo}& \multicolumn{4}{m{11cm}|}{\justifying Problemas de compatibilidad de hardware.}\\\hline
    {\bf Prioridad}& 3 & {\bf Impacto}& 4 & {\bf Probabilidad}& 0.9\\\hline
    \multicolumn{2}{|m{5cm}}{\bf Tipo de Riesgo}& \multicolumn{4}{m{11cm}|}{\justifying T\'ecnico.}\\\hline
    \multicolumn{2}{|m{5cm}}{\bf Contexto}& \multicolumn{4}{m{11cm}|}{\justifying El hardware del celular no es el esperado, y no presenta todas las caracter\'isticas requeridas para el correcto funcionamiento de la aplicaci\'on, por ejemplo se quiere utilizar el sensor de luz de un smartphone para cierta funci\'on de la aplicaci\'on, pero el Smartphone no lo tiene, por lo que el programa se cae.}\\\hline
    \multicolumn{2}{|m{5cm}}{\bf Plan de Contingencia}& \multicolumn{4}{m{11cm}|}{\justifying Al tratar de utilizar un sensor que no esta presente en el smartphone puede provocar la ca\'ida de la aplicaci\'on, por lo cual esta no ser\'ia compatible con todos los smartphones.\\
Si no se puede detectar la presencia de sensores en el smartphone se utilizaran los m\'as comunes presentes en los smartphone.}\tabularnewline\hline
    \multicolumn{2}{|m{5cm}}{\bf Plan de Mitigaci\'on}& \multicolumn{4}{m{11cm}|}{\justifying De no estar presente alguno de los sensores que se desea utilizar:\begin{enumerate}\item Detectar los sensores presentes en el smartphone  para bloquear o dar opciones acerca de su uso.\item Disminuir el uso de sensores al m\'inimo.\end{enumerate}}\\\hline
    \multicolumn{2}{|m{5cm}}{\bf Resoluci\'on}& \multicolumn{4}{m{11cm}|}{\justifying Al mitigar este riesgo la aplicaci\'on no se caer\'a cuando intente usar sensores del smartphone.}\\\hline
  \end{tabular}
  \caption[~RT2]{}
  \label{table:RT2}
\end{table}


%%
%%Riesgo RT3
%%
\begin{table}[htbp!]
  \centering
  \begin{tabular}{|p{4cm}p{1cm}|p{4cm}p{1cm}|p{4cm}p{1cm}|}\hline
        \multicolumn{2}{|m{5cm}}{\bf Riesgo}& \multicolumn{4}{m{11cm}|}{\justifying Inconsistencia entre robot f\'isico y mascota virtual.}\\\hline
    {\bf Prioridad}& 4 & {\bf Impacto}& 4 & {\bf Probabilidad}& 0.8\\\hline
    \multicolumn{2}{|m{5cm}}{\bf Tipo de Riesgo}& \multicolumn{4}{m{11cm}|}{\justifying T\'ecnico.}\\\hline
    \multicolumn{2}{|m{5cm}}{\bf Contexto}& \multicolumn{4}{m{11cm}|}{\justifying La arquitectura f\'isica es distinta a la de la figura virtual, por ejemplo la mascota virtual considera tres motores pero el robot f\'isico presenta solo dos, otro caso podr\'ia ser que los motores y sensores est\'an conectados en puertos distintos a los que considera el programa, resultando en el mal funcionamiento del robot y la aplicaci\'on.}\\\hline
    \multicolumn{2}{|m{5cm}}{\bf Plan de Contingencia}& \multicolumn{4}{m{11cm}|}{\justifying Al generar una reacci\'on en el robot por medio de la aplicaci\'on este no reaccionaria de la forma esperada, por ejemplo, se quiere que el robot camine, pero los motores de sus patas no est\'an conectados donde la aplicaci\'on espera, lo que va a resultar en que el perro no pueda caminar correctamente.\\
De darse esto, podr\'ia advertirse al usuario que el robot no se puede modificar, o si lo hace es bajo su propio riesgo.}\tabularnewline\hline
    \multicolumn{2}{|m{5cm}}{\bf Plan de Mitigaci\'on}& \multicolumn{4}{m{11cm}|}{\justifying Para evitar problemas de inconsistencia entre la maqueta f\'isica y virtual de la mascota se puede:\begin{enumerate}\item Crear un wizard de configuraci\'on de los motores, de manera que el usuario pueda modificar el robot sin problemas.\item Crear un mensaje donde se indique la posicion en donde el programa espera que esten conectados los motores.\end{enumerate}}\\\hline
    \multicolumn{2}{|m{5cm}}{\bf Resoluci\'on}& \multicolumn{4}{m{11cm}|}{\justifying Al ejecutar una acci\'on en el robot la cual proviene de la aplicaci\'on este la realizar\'a sin problemas.}\\\hline
  \end{tabular}
  \caption[~RT3]{}
  \label{table:RT3}
\end{table}


%%
%%Riesgo RT4
%%
\begin{table}[htbp!]
  \centering
  \begin{tabular}{|p{4cm}p{1cm}|p{4cm}p{1cm}|p{4cm}p{1cm}|}\hline
        \multicolumn{2}{|m{5cm}}{\bf Riesgo}& \multicolumn{4}{m{11cm}|}{\justifying P\'erdida de acceso al robot, ya sea por robo o falla t\'ecnica.}\\\hline
    {\bf Prioridad}& 4 & {\bf Impacto}& 5 & {\bf Probabilidad}& 0.3\\\hline
    \multicolumn{2}{|m{5cm}}{\bf Tipo de Riesgo}& \multicolumn{4}{m{11cm}|}{\justifying T\'ecnico.}\\\hline
    \multicolumn{2}{|m{5cm}}{\bf Contexto}& \multicolumn{4}{m{11cm}|}{\justifying Para la realizaci\'on del proyecto es necesaria la utilizaci\'on del NXT Intelligent Brick y componentes de este como motores y sensores, la falla de cualquiera de estos o la p\'erdida de acceso debido a robo pueden causar serios problemas con el avance del proyecto llev\'andolo al fracaso.}\\\hline
    \multicolumn{2}{|m{5cm}}{\bf Plan de Contingencia}& \multicolumn{4}{m{11cm}|}{\justifying Se requiere el robot para la interacci\'on con la aplicaci\'on, sin la presencia de este no se puede avanzar el proyecto o se deber\'a perder gran parte de la innovaci\'on de este.\\
De no tener acceso al robot se deber\'a eliminar toda interacci\'on con este.}\tabularnewline\hline
    \multicolumn{2}{|m{5cm}}{\bf Plan de Mitigaci\'on}& \multicolumn{4}{m{11cm}|}{\justifying En caso de p\'erdida, robo o falla del robot se puede:\begin{enumerate}\item Comprar la pieza que falla.\item Pedir un robot nuevo a la Universidad o Cliente.\item Comprar un kit NXT Intelligent Brick nuevo.\end{enumerate}}\\\hline
    \multicolumn{2}{|m{5cm}}{\bf Resoluci\'on}& \multicolumn{4}{m{11cm}|}{\justifying Si se tiene el robot funcionando $100\%$ se podr\'a avanzar en el proyecto sin retrasos.}\\\hline
  \end{tabular}
  \caption[~RT4]{}
  \label{table:RT4}
\end{table}

\newpage
\subsection{Riesgos de Proyecto}
%%
%%Riesgo RP1
%%
\begin{table}[htbp!]
  \centering
  \begin{tabular}{|p{4cm}p{1cm}|p{4cm}p{1cm}|p{4cm}p{1cm}|}\hline
        \multicolumn{2}{|m{5cm}}{\bf Riesgo}& \multicolumn{4}{m{11cm}|}{\justifying Falta de experiencia de miembros del equipo en programaci\'on en ROBOTC y Android.}\\\hline
    {\bf Prioridad}& 3 & {\bf Impacto}& 4 & {\bf Probabilidad}& 0.6\\\hline
    \multicolumn{2}{|m{5cm}}{\bf Tipo de Riesgo}& \multicolumn{4}{m{11cm}|}{\justifying Proyecto.}\\\hline
    \multicolumn{2}{|m{5cm}}{\bf Contexto}& \multicolumn{4}{m{11cm}|}{\justifying Este proyecto requiere programaci\'on en Android SDK el cual trabaja con Java, y para el NXT Intelligent Brick ROBOTC, por esta raz\'on es necesario tener conocimientos de estos lenguajes, para el correcto avance del proyecto.}\\\hline
    \multicolumn{2}{|m{5cm}}{\bf Plan de Contingencia}& \multicolumn{4}{m{11cm}|}{\justifying Si gran parte de los miembros del equipo de trabajo no cuentan con conocimientos en los lenguajes que se requiere utilizar, esto puede a llevar a que el proyecto falle.\\
Por esto como medida de emergencia se recargar\'a a los miembros del equipo con m\'as conocimiento en los lenguajes a utilizar.}\tabularnewline\hline
    \multicolumn{2}{|m{5cm}}{\bf Plan de Mitigaci\'on}& \multicolumn{4}{m{11cm}|}{\justifying En caso de que los miembros del equipo no cuenten con suficiente conocimiento y experiencia se puede:\begin{enumerate}\item Capacitar a los miembros del equipo con menos conocimientos recibiendo clases de parte de los miembros con m\'as conocimientos.\item Utilizar la documentaci\'on disponible en internet como herramienta de autoaprendizaje.\end{enumerate}}\\\hline
    \multicolumn{2}{|m{5cm}}{\bf Resoluci\'on}& \multicolumn{4}{m{11cm}|}{\justifying Si los miembros del equipo tiene conocimiento y experiencia con los lenguajes que se requiere utilizar disminuyen dr\'asticamente las probabilidades de que el proyecto falle.}\\\hline
  \end{tabular}
  \caption[~RP1]{}
  \label{table:RP1}
\end{table}


%%
%%Riesgo RP2
%%
\begin{table}[htbp!]
  \centering
  \begin{tabular}{|p{4cm}p{1cm}|p{4cm}p{1cm}|p{4cm}p{1cm}|}\hline
        \multicolumn{2}{|m{5cm}}{\bf Riesgo}& \multicolumn{4}{m{11cm}|}{\justifying P\'erdida de personal.}\\\hline
    {\bf Prioridad}& 2 & {\bf Impacto}& 4 & {\bf Probabilidad}& 0.9\\\hline
    \multicolumn{2}{|m{5cm}}{\bf Tipo de Riesgo}& \multicolumn{4}{m{11cm}|}{\justifying Proyecto.}\\\hline
    \multicolumn{2}{|m{5cm}}{\bf Contexto}& \multicolumn{4}{m{11cm}|}{\justifying El equipo cuenta con 5 miembros los cuales se dividen el trabajo para avanzar en el proyecto, y presentar las entregas en las fechas requeridas. El tama\~no del equipo es adecuado para el trabajo que debe realizarse, pero pueden ocurrir casos donde uno o m\'as miembros no puedan colaborar con el trabajo ya se a  causa de una enfermedad, problemas personales que generen poca disponibilidad o falta de tiempo a causa de otros ramos.}\\\hline
    \multicolumn{2}{|m{5cm}}{\bf Plan de Contingencia}& \multicolumn{4}{m{11cm}|}{\justifying Se deben realizar entregas peri\'odicas de avance, ya sean informes como avance de la aplicaci\'on, por lo que se debe trabajar en conjunto para tener las entregas listas en la fecha que se requiere.\\
En el caso de que alg\'un miembro no pueda realizar el trabajo que se le hab\'ia asignado el equipo se ver\'a en la obligaci\'on de repartir su parte entre los miembros restantes.}\tabularnewline\hline
    \multicolumn{2}{|m{5cm}}{\bf Plan de Mitigaci\'on}& \multicolumn{4}{m{11cm}|}{\justifying Si uno o m\'as de los miembros de equipo no puede realizar su trabajo se puede:\begin{enumerate}\item En caso de que los miembros no puedan tener disponibilidad debido a pruebas de otros ramos se puede realizar un calendario con todas las evaluaciones y as\'i poder programar c\'omo y cu\'ando se trabajar\'a.\item En caso de enfermedad, se puede disminuir la carga de trabajo del individuo enfermo o redistribuir los trabajos para, en caso de que se pueda, permitir que trabaje desde su hogar.\item En caso de p\'erdida definitiva se disminuir\'a el alcance del proyecto.\end{enumerate}}\\\hline
    \multicolumn{2}{|m{5cm}}{\bf Resoluci\'on}& \multicolumn{4}{m{11cm}|}{\justifying Al tener disponibilidad de todo el personal de trabajo no solo existe una mayor probabilidad de entregar en la fecha adecuada, sino que tambi\'en se disminuye la carga de todos.}\\\hline
  \end{tabular}
  \caption[~RP2]{}
  \label{table:RP2}
\end{table}


%%
%%Riesgo RP3
%%
\begin{table}[htbp!]
  \centering
  \begin{tabular}{|p{4cm}p{1cm}|p{4cm}p{1cm}|p{4cm}p{1cm}|}\hline
        \multicolumn{2}{|m{5cm}}{\bf Riesgo}& \multicolumn{4}{m{11cm}|}{\justifying Problemas inesperados durante la implementaci\'on del proyecto.}\\\hline
    {\bf Prioridad}& 3 & {\bf Impacto}& 2 & {\bf Probabilidad}& 0.2\\\hline
    \multicolumn{2}{|m{5cm}}{\bf Tipo de Riesgo}& \multicolumn{4}{m{11cm}|}{\justifying Proyecto.}\\\hline
    \multicolumn{2}{|m{5cm}}{\bf Contexto}& \multicolumn{4}{m{11cm}|}{\justifying Al ser un proyecto en el cual se trabaja con herramientas nuevas, siempre existe la posibilidad de tener problemas inesperados en su transcurso, los cuales pueden afectar directamente su probabilidad de \'exito.\\
En el caso de que ocurran problemas inesperados que consuman tiempo extra no esperado al trabajar, por ejemplo, con  Android y ROBOTC.}\tabularnewline\hline
    \multicolumn{2}{|m{5cm}}{\bf Plan de Contingencia}& \multicolumn{4}{m{11cm}|}{\justifying Si llegasen a ocurrir errores al trabajar con Android, ROBOTC o alguna de las herramientas que se requieren usar y consuman demasiado tiempo afectando el avance en las entregas se  deber\'a encontrar la medida de contingencia adecuada, siendo la persona m\'as experimentada en el \'area del problema la encargada de dirigir el proceso. El resto del equipo validar\'a e implementar\'a la soluci\'on una vez encontrada.}\\\hline
    \multicolumn{2}{|m{5cm}}{\bf Plan de Mitigaci\'on}& \multicolumn{4}{m{11cm}|}{\justifying En caso de que problemas inesperados ocurran durante la implementaci\'on  del proyecto se puede:\begin{enumerate}\item Organiza el calendario de tal forma de dejar un margen entre la fecha de entrega y la finalizaci\'on de \'esta, para as\'i evitar casos de falta de tiempo por problemas inesperados.\item Volver a repartir el trabajo entre el equipo para ayudar a quien se vea afectado por el problema.\end{enumerate}}\\\hline
    \multicolumn{2}{|m{5cm}}{\bf Resoluci\'on}& \multicolumn{4}{m{11cm}|}{\justifying Si se pueden prevenir los problemas inesperados, se evitar\'an problemas de falta de tiempo al realizar las entregas.}\\\hline
  \end{tabular}
  \caption[~RP3]{}
  \label{table:RP3}
\end{table}


%%
%%Riesgo RP4
%%
\begin{table}[htbp!]
  \centering
  \begin{tabular}{|p{4cm}p{1cm}|p{4cm}p{1cm}|p{4cm}p{1cm}|}\hline
        \multicolumn{2}{|m{5cm}}{\bf Riesgo}& \multicolumn{4}{m{11cm}|}{\justifying Alto costo monetario de hardware o software necesario para la implementaci\'on.}\\\hline
    {\bf Prioridad}& 3 & {\bf Impacto}& 3 & {\bf Probabilidad}& 0.5\\\hline
    \multicolumn{2}{|m{5cm}}{\bf Tipo de Riesgo}& \multicolumn{4}{m{11cm}|}{\justifying Proyecto.}\\\hline
    \multicolumn{2}{|m{5cm}}{\bf Contexto}& \multicolumn{4}{m{11cm}|}{\justifying Se trabajar\'a con smartphones y LEGO Mindstorms para la realizaci\'on del proyecto, por lo que se pueden requerir componentes que no estaban presupuestados o tengan un valor mayor  al esperado, por ejemplo, sensores o piezas para el robot.\\
Adem\'as se requiere licencias para software, por ejemplo, ROBOTC.}\tabularnewline\hline
    \multicolumn{2}{|m{5cm}}{\bf Plan de Contingencia}& \multicolumn{4}{m{11cm}|}{\justifying Dado que muchos de los elementos requeridos para el proyecto, ya sea hardware o software conllevan un costo, es probable que est\'en fuera del presupuesto esperado, lo cual afectar\'ia directamente en el proyecto.\\
En caso de no poder costear alguno de los elementos necesarios se deber\'a obviar su uso, esto dependiendo de qu\'e tan esencial sea para el \'exito del proyecto.}\tabularnewline\hline
    \multicolumn{2}{|m{5cm}}{\bf Plan de Mitigaci\'on}& \multicolumn{4}{m{11cm}|}{\justifying De no contar con presupuesto para tener acceso a hardware o software necesario se puede:\begin{enumerate}\item Solicitar al cliente que cubra los gastos necesarios.\item Preparar un saldo de emergencia para casos como este.\end{enumerate}}\\\hline
    \multicolumn{2}{|m{5cm}}{\bf Resoluci\'on}& \multicolumn{4}{m{11cm}|}{\justifying Al tener acceso a todos los elementos necesario, ya sean hardware o software disminuir\'a las probabilidades de fallo del proyecto.}\\\hline
  \end{tabular}
  \caption[~RP4]{}
  \label{table:RP4}
\end{table}

\newpage
\subsection{Riesgos de Negocio}
%%
%%Riesgo RN1
%%
\begin{table}[htbp!]
  \centering
  \begin{tabular}{|p{4cm}p{1cm}|p{4cm}p{1cm}|p{4cm}p{1cm}|}\hline
        \multicolumn{2}{|m{5cm}}{\bf Riesgo}& \multicolumn{4}{m{11cm}|}{\justifying Cese y desista por parte de LEGO.}\\\hline
    {\bf Prioridad}& 1 & {\bf Impacto}& 5 & {\bf Probabilidad}& 0.1\\\hline
    \multicolumn{2}{|m{5cm}}{\bf Tipo de Riesgo}& \multicolumn{4}{m{11cm}|}{\justifying Negocio.}\\\hline
    \multicolumn{2}{|m{5cm}}{\bf Contexto}& \multicolumn{4}{m{11cm}|}{\justifying Para la realizaci\'on del proyecto  se utilizar\'a LEGO Mindstorms para el dise\~no del robot con el cual la aplicaci\'on va a interactuar, por lo cual es importante cumplir con todos los t\'erminos y condiciones de este para as\'i no tener problemas de tipo legales con la empresa.}\\\hline
    \multicolumn{2}{|m{5cm}}{\bf Plan de Contingencia}& \multicolumn{4}{m{11cm}|}{\justifying En el caso de que LEGO denegar\'a los permisos para utilizar LEGO Mindstorms por cualquier incumplimiento de los t\'erminos y condiciones de uso, esto afectar\'ia una parte importante del proyecto quitando gran parte de la innovaci\'on de este, y en el peor caso obligar\'ia a cancelar este.}\\\hline
    \multicolumn{2}{|m{5cm}}{\bf Plan de Mitigaci\'on}& \multicolumn{4}{m{11cm}|}{\justifying En caso de no tener permisos para utilizar LEGO Mindstorms se puede:\begin{enumerate}\item Crear el robot con otro tipo de tecnolog\'ias, por ejemplo, Arduino.\end{enumerate}}\\\hline
    \multicolumn{2}{|m{5cm}}{\bf Resoluci\'on}& \multicolumn{4}{m{11cm}|}{\justifying Si se tiene permisos para utilizar LEGO Mindstorms para el dise\~no del robot se podr\'a seguir contando con la innovaci\'on del negocio.}\\\hline
  \end{tabular}
  \caption[~RN1]{}
  \label{table:RN1}
\end{table}



%%
%%Riesgo RN2
%%
\begin{table}[htbp!]
  \centering
  \begin{tabular}{|p{4cm}p{1cm}|p{4cm}p{1cm}|p{4cm}p{1cm}|}\hline
        \multicolumn{2}{|m{5cm}}{\bf Riesgo}& \multicolumn{4}{m{11cm}|}{\justifying Aplicaci\'on poco atractiva para p\'ublico objetivo.}\\\hline
    {\bf Prioridad}& 1 & {\bf Impacto}& 2 & {\bf Probabilidad}& 0.9\\\hline
    \multicolumn{2}{|m{5cm}}{\bf Tipo de Riesgo}& \multicolumn{4}{m{11cm}|}{\justifying Negocio.}\\\hline
    \multicolumn{2}{|m{5cm}}{\bf Contexto}& \multicolumn{4}{m{11cm}|}{\justifying La aplicaci\'on esta enfocada a escolares de ense\~nanza media, por lo que esta debe ser atractiva, para as\'i poder cumplir con el objetivo de acercarlos a la inform\'atica.\\
Por esto la aplicaci\'on debe poder atraer la atenci\'on los usuarios como as\'i tambi\'en lograr generar un inter\'es en la inform\'atica.}\tabularnewline\hline
    \multicolumn{2}{|m{5cm}}{\bf Plan de Contingencia}& \multicolumn{4}{m{11cm}|}{\justifying Al tratarse de una aplicaci\'on que tiene como p\'ublico objetivo adolescentes, esta adem\'as de tener una apariencia agradable para ellos debe entretenerlos, si esto no se logra har\'a m\'as dif\'icil cumplir el objetivo de la aplicaci\'on.\\
En el  caso de que la interfaz y caracter\'isticas de la aplicaci\'on no sean atractivas para el p\'ublico objetivo se deber\'a en base a encuestas y pruebas con usuarios trabajar sobre las caracter\'isticas actuales aunque esto implique cambiar gran parte de la aplicaci\'on y su funcionamiento.}\\\hline
    \multicolumn{2}{|m{5cm}}{\bf Plan de Mitigaci\'on}& \multicolumn{4}{m{11cm}|}{\justifying En caso de que la aplicaci\'on no resulte atractiva para el p\'ublico objetivo se puede:\begin{enumerate}\item Realizar un testing con un grupo de estudiantes y obtener datos.\item Realizar encuestas a un grupo de usuarios objetivos.\end{enumerate}}\\\hline
    \multicolumn{2}{|m{5cm}}{\bf Resoluci\'on}& \multicolumn{4}{m{11cm}|}{\justifying Al tener una aplicaci\'on atractiva para los estudiantes se podr\'a cumplir m\'as f\'acilmente el objetivo del negocio.}\\\hline
  \end{tabular}
  \caption[~~RN2]{}
  \label{table:RN2}
\end{table}
