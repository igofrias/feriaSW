\section{Gesti\'on de riesgos.}

\subsection{Análisis de riesgos.}

\begin{itemize}
\item {\bf Riesgos Técnicos}:
  \begin{itemize}
  \item[R1.] Errores de inicio y mantención de conexión vía Bluetooth entre aplicación y sistema robótico.
  \item[R2.] Problemas de compatibilidad de hardware.
  \item[R3.] Inconsistencia entre robot físico y mascota virtual.
  \item[R4.] Pérdida de acceso al robot, ya sea por robo o falla técnica.
  \end{itemize}
\item {\bf Riesgos de Proyecto}:
  \begin{itemize}
  \item[R5.] Falta de experiencia de miembros del equipo en programación en ROBOTC y Android
  \item[R6.] Pérdida de personal.
  \item[R7.] Problemas inesperados durante la implementación del proyecto.
  \item[R8.] Alto costo monetario de hardware o software necesario para la implementación.
  \end{itemize}
\item {\bf Riesgos de Negocio}:
  \begin{itemize}
  \item[R9.] Cese y desista por parte de LEGO.
  \item[R10.] Aplicación poco atractiva para público objetivo.
  \end{itemize}
\end{itemize}

\subsection{Preparación para control de riesgos.}

\subsection{Riesgos T\'ecnicos}

%%
%%Riesgo R1
%%
\begin{table}[H]
  \centering
  \begin{tabular}{|p{15cm}|}\hline
    {\bf Riesgo}: Errores de inicio y mantención de conexión vía Bluetooth entre aplicación y sistema robótico.\\\hline
    {\bf Prioridad}: 5  {\bf Impacto}: 5  {\bf Probabilidad}: 0.3\\\hline
    {\bf Tipo de Riesgo}: Técnico.\\\hline
    {\bf Contexto}: La interacción de ambos sistemas se da exclusivamente por este medio, lo cual influye de manera considerable en la funcionalidad que se pueda obtener.\\Se estima esencial la conexión entre el robot y el smartphone ya que la correlación de ambas herramientas, así como la fuente de su innovación, reside en su comunicación.\\\hline
    {\bf Plan de Contingencia}: El control de este riesgo es fundamental para el proyecto, ya sea lograr una conexión exitosa, como mantenerla durante el tiempo de utilización de la aplicación.\\En caso de no poder cumplir uno de estos dos requerimientos se deberán tomar medidas para mitigar el problema, de no poder reemplazar la conexión, se trabajara solo con uno de los dos aparatos (NXT o Smarthphone) lo que disminuye notablemente la innovación del proyecto.\\\hline
    {\bf Plan de Mitigación}: De no lograrse una conexión exitosa entre el NXT Intelligent Brick y el smartphone con Android:\begin{enumerate}\item se utilizará un aparato móvil con sistema operativo iOS permitiendo conservar la fuente de innovación tanto por parte de la interacción entre el dispositivo y el robot, como la utilización de sensores del smartphone para variadas funcionalidades.\item Se realizará una conexión vía bluetooth con un computador.\item Se realizará una conexión vía USB con el computador.\end{enumerate}\\\hline
    {\bf Resolución}: Al conectar exitosamente los dispositivos y mantener la conexión se tendrá la base para llevar a cabo todo lo que demanda el proyecto.\\\hline
  \end{tabular}
  \label{table:R1}
\end{table}


%%
%%Riesgo R2
%%
\begin{table}[H]
  \centering
  \begin{tabular}{|p{15cm}|}\hline
    {\bf Riesgo}: Problemas de compatibilidad de hardware.\\\hline
    {\bf Prioridad}: 3  {\bf Impacto}: 4  {\bf Probabilidad}: 0.9\\\hline
    {\bf Tipo de Riesgo}: Técnico\\\hline
    {\bf Contexto}: El hardware del celular no es el esperado, y no presenta todas las características requeridas para el correcto funcionamiento de la aplicación, por ejemplo se quiere utilizar el sensor de luz de un smartphone para cierta función de la aplicación, pero el Smartphone no lo tiene, por lo que el programa se cae.\\\hline
    {\bf Plan de Contingencia}: Al tratar de utilizar un sensor que no esta presente en el smartphone puede provocar la caída de la aplicación, por lo cual esta no sería compatible con todos los smartphones.\\Si no se puede detectar la presencia de sensores en el smartphone se utilizaran los m\'as comunes presentes en los smartphone.\\\hline
    {\bf Plan de Mitigación}: De no estar presente alguno de los sensores que se desea utilizar:\begin{enumerate}\item Detectar los sensores presentes en el smartphone  para bloquear o dar opciones acerca de su uso.\item Disminuir el uso de sensores al mínimo.\end{enumerate}\\\hline
    {\bf Resolución}: Al mitigar este riesgo la aplicación no se caerá cuando intente usar sensores del smartphone.\\\hline
  \end{tabular}
  \label{table:R2}
\end{table}


%%
%%Riesgo R3
%%
\begin{table}[H]
  \centering
  \begin{tabular}{|p{15cm}|}\hline
    {\bf Riesgo}: Inconsistencia entre robot físico y mascota virtual.\\\hline
    {\bf Prioridad}: 4  {\bf Impacto}: 4  {\bf Probabilidad}: 0.8\\\hline
    {\bf Tipo de Riesgo}: Técnico.\\\hline
    {\bf Contexto}: La arquitectura física es distinta a la de la figura virtual, por ejemplo la mascota virtual considera tres motores pero el robot físico presenta solo dos, otro caso podría ser que los motores y sensores están conectados en puertos distintos a los que considera el programa, resultando en el mal funcionamiento del robot y la aplicación.\\\hline
    {\bf Plan de Contingencia}: Al generar una reacción en el robot por medio de la aplicación este no reaccionaria de la forma esperada, por ejemplo, se quiere que el robot camine, pero los motores de sus patas no están conectados donde la aplicación espera, lo que va a resultar en que el perro no pueda caminar correctamente.\\De darse esto, podría advertirse al usuario que el robot no se puede modificar, o si lo hace es bajo su propio riesgo.\\\hline
    {\bf Plan de Mitigación}: Para evitar problemas de inconsistencia entre la maqueta física y virtual de la mascota se puede:\begin{enumerate}\item Crear un wizard de configuración de los motores, de manera que el usuario pueda modificar el robot sin problemas.\item Crear un mensaje donde se indique la posicion en donde el programa espera que esten conectados los motores.\end{enumerate}\\\hline
    {\bf Resolución}: Al ejecutar una acción en el robot la cual proviene de la aplicación este la realizará sin problemas.\\\hline
  \end{tabular}
  \label{table:R3}
\end{table}


%%
%%Riesgo R4
%%
\begin{table}[H]
  \centering
  \begin{tabular}{|p{15cm}|}\hline
    {\bf Riesgo}: Pérdida de acceso al robot, ya sea por robo o falla técnica.\\\hline
    {\bf Prioridad}: 4  {\bf Impacto}: 5  {\bf Probabilidad}: 0.3\\\hline
    {\bf Tipo de Riesgo}: Técnico.\\\hline
    {\bf Contexto}: Para la realización del proyecto es necesaria la utilización del NXT Intelligent Brick y componentes de este como motores y sensores, la falla de cualquiera de estos o la pérdida de acceso debido a robo pueden causar serios problemas con el avance del proyecto llevándolo al fracaso.\\\hline
    {\bf Plan de Contingencia}: Se requiere el robot para la interacción con la aplicación, sin la presencia de este no se puede avanzar el proyecto o se deberá perder gran parte de la innovación de este.\\De no tener acceso al robot se deberá eliminar toda interacción con este.\\\hline
    {\bf Plan de Mitigación}: En caso de pérdida, robo o falla del robot se puede:\begin{enumerate}\item Comprar la pieza que falla.\item Pedir un robot nuevo a la Universidad o Cliente.\item Comprar un kit NXT Intelligent Brick nuevo.\end{enumerate}\\\hline
    {\bf Resolución}: Si se tiene el robot funcionando $100\%$ se podrá avanzar en el proyecto sin retrasos.\\\hline
  \end{tabular}
  \label{table:R4}
\end{table}


%%
%%Riesgo R5
%%
\begin{table}[H]
  \centering
  \begin{tabular}{|p{15cm}|}\hline
    {\bf Riesgo}: Falta de experiencia de miembros del equipo en programación en ROBOTC y Android.\\\hline
    {\bf Prioridad}: 3  {\bf Impacto}: 4  {\bf Probabilidad}: 0.6\\\hline
    {\bf Tipo de Riesgo}: Proyecto.\\\hline
    {\bf Contexto}: Este proyecto requiere programación en Android SDK el cual trabaja con Java, y para el NXT Intelligent Brick ROBOTC, por esta razón es necesario tener conocimientos de estos lenguajes, para el correcto avance del proyecto.\\\hline
    {\bf Plan de Contingencia}: Si gran parte de los miembros del equipo de trabajo no cuentan con conocimientos en los lenguajes que se requiere utilizar, esto puede a llevar a que el proyecto falle.\\Por esto como medida de emergencia se recargará a los miembros del equipo con más conocimiento en los lenguajes a utilizar.\\\hline
    {\bf Plan de Mitigación}: En caso de que los miembros del equipo no cuenten con suficiente conocimiento y experiencia se puede:\begin{enumerate}\item Capacitar a los miembros del equipo con menos conocimientos recibiendo clases de parte de los miembros con más conocimientos.\item Utilizar la documentación disponible en internet como herramienta de autoaprendizaje.\end{enumerate}\\\hline
    {\bf Resolución}: Si los miembros del equipo tiene conocimiento y experiencia con los lenguajes que se requiere utilizar disminuyen drásticamente las probabilidades de que el proyecto falle.\\\hline
  \end{tabular}
  \label{table:R5}
\end{table}


%%
%%Riesgo R6
%%
\begin{table}[H]
  \centering
  \begin{tabular}{|p{15cm}|}\hline
    {\bf Riesgo}: Pérdida de personal.\\\hline
    {\bf Prioridad}: 2  {\bf Impacto}: 4  {\bf Probabilidad}: 0.9\\\hline
    {\bf Tipo de Riesgo}: Proyecto.\\\hline
    {\bf Contexto}: El equipo cuenta con 5 miembros los cuales se dividen el trabajo para avanzar en el proyecto, y presentar las entregas en las fechas requeridas. El tamaño del equipo es adecuado para el trabajo que debe realizarse, pero pueden ocurrir casos donde uno o más miembros no puedan colaborar con el trabajo ya se a  causa de una enfermedad, problemas personales que generen poca disponibilidad o falta de tiempo a causa de otros ramos.\\\hline
    {\bf Plan de Contingencia}: Se deben realizar entregas periódicas de avance, ya sean informes como avance de la aplicación, por lo que se debe trabajar en conjunto para tener las entregas listas en la fecha que se requiere.\\En el caso de que algún miembro no pueda realizar el trabajo que se le había asignado el equipo se verá en la obligación de repartir su parte entre los miembros restantes.\\\hline
    {\bf Plan de Mitigación}: Si uno o más de los miembros de equipo no puede realizar su trabajo se puede:\begin{enumerate}\item En caso de que los miembros no puedan tener disponibilidad debido a pruebas de otros ramos se puede realizar un calendario con todas las evaluaciones y así poder programar cómo y cuándo se trabajará.\item En caso de enfermedad, se puede disminuir la carga de trabajo del individuo enfermo o redistribuir los trabajos para, en caso de que se pueda, permitir que trabaje desde su hogar.\item En caso de pérdida definitiva se disminuirá el alcance del proyecto.\end{enumerate}\\\hline
    {\bf Resolución}: Al tener disponibilidad de todo el personal de trabajo no solo existe una mayor probabilidad de entregar en la fecha adecuada, sino que también se disminuye la carga de todos.\\\hline
  \end{tabular}
  \label{table:R6}
\end{table}


%%
%%Riesgo R7
%%
\begin{table}[H]
  \centering
  \begin{tabular}{|p{15cm}|}\hline
    {\bf Riesgo}: Problemas inesperados durante la implementación del proyecto.\\\hline
    {\bf Prioridad}: 3  {\bf Impacto}: 2  {\bf Probabilidad}: 0.2\\\hline
    {\bf Tipo de Riesgo}: Proyecto.\\\hline
    {\bf Contexto}: Al ser un proyecto en el cual se trabaja con herramientas nuevas, siempre existe la posibilidad de tener problemas inesperados en su transcurso, los cuales pueden afectar directamente su probabilidad de éxito.\\En el caso de que ocurran problemas inesperados que consuman tiempo extra no esperado al trabajar, por ejemplo, con  Android y ROBOTC.\\\hline
    {\bf Plan de Contingencia}: Si llegasen a ocurrir errores al trabajar con Android, ROBOTC o alguna de las herramientas que se requieren usar y consuman demasiado tiempo afectando el avance en las entregas se  deberá encontrar la medida de contingencia adecuada, siendo la persona más experimentada en el área del problema la encargada de dirigir el proceso. El resto del equipo validará e implementará la solución una vez encontrada.\\\hline
    {\bf Plan de Mitigación}:  En caso de que problemas inesperados ocurran durante la implementación  del proyecto se puede:\begin{enumerate}\item Organiza el calendario de tal forma de dejar un margen entre la fecha de entrega y la finalización de ésta, para así evitar casos de falta de tiempo por problemas inesperados.\item Volver a repartir el trabajo entre el equipo para ayudar a quien se vea afectado por el problema.\end{enumerate}\\\hline
    {\bf Resolución}: Si se pueden prevenir los problemas inesperados, se evitarán problemas de falta de tiempo al realizar las entregas.\\\hline
  \end{tabular}
  \label{table:R7}
\end{table}


%%
%%Riesgo R8
%%
\begin{table}[H]
  \centering
  \begin{tabular}{|p{15cm}|}\hline
    {\bf Riesgo}: Alto costo monetario de hardware o software necesario para la implementación.\\\hline
    {\bf Prioridad}: 3  {\bf Impacto}: 3  {\bf Probabilidad}: 0.5\\\hline
    {\bf Tipo de Riesgo}: Proyecto.\\\hline
    {\bf Contexto}: Se trabajará con smartphones y LEGO Mindstorms para la realización del proyecto, por lo que se pueden requerir componentes que no estaban presupuestados o tengan un valor mayor  al esperado, por ejemplo, sensores o piezas para el robot.\\Además se requiere licencias para software, por ejemplo, ROBOTC.\\\hline
    {\bf Plan de Contingencia}: Dado que muchos de los elementos requeridos para el proyecto, ya sea hardware o software conllevan un costo, es probable que estén fuera del presupuesto esperado, lo cual afectaría directamente en el proyecto.\\En caso de no poder costear alguno de los elementos necesarios se deberá obviar su uso, esto dependiendo de qué tan esencial sea para el éxito del proyecto.\\\hline
    {\bf Plan de Mitigación}: De no contar con presupuesto para tener acceso a hardware o software necesario se puede:\begin{enumerate}\item Solicitar al cliente que cubra los gastos necesarios.\item Preparar un saldo de emergencia para casos como este.\end{enumerate}\\\hline
    {\bf Resolución}: Al tener acceso a todos los elementos necesario, ya sean hardware o software disminuirá las probabilidades de fallo del proyecto.\\\hline
  \end{tabular}
  \label{table:R8}
\end{table}


%%
%%Riesgo R9
%%
\begin{table}[H]
  \centering
  \begin{tabular}{|p{15cm}|}\hline
    {\bf Riesgo}: Aplicación poco atractiva para público objetivo.\\\hline
    {\bf Prioridad}: 1  {\bf Impacto}: 2  {\bf Probabilidad}: 0.9\\\hline
    {\bf Tipo de Riesgo}: Negocio.\\\hline
    {\bf Contexto}: La aplicación esta enfocada a escolares de enseñanza media, por lo que esta debe ser atractiva, para así poder cumplir con el objetivo de acercarlos a la informática.\\Por esto la aplicación debe poder atraer la atención los usuarios como así también lograr generar un interés en la informática.\\\hline
    {\bf Plan de Contingencia}: Al tratarse de una aplicación que tiene como público objetivo adolescentes, esta además de tener una apariencia agradable para ellos debe entretenerlos, si esto no se logra hará más difícil cumplir el objetivo de la aplicación.\\En el  caso de que la interfaz y características de la aplicación no sean atractivas para el público objetivo se deberá en base a encuestas y pruebas con usuarios trabajar sobre las características actuales aunque esto implique cambiar gran parte de la aplicación y su funcionamiento.\\\hline
    {\bf Plan de Mitigación}: En caso de que la aplicación no resulte atractiva para el público objetivo se puede:\begin{enumerate}\item Realizar un testing con un grupo de estudiantes y obtener datos.\item Realizar encuestas a un grupo de usuarios objetivos.\end{enumerate}\\\hline
    {\bf Resolución}: Al tener una aplicación atractiva para los estudiantes se podrá cumplir más fácilmente el objetivo del negocio.
  \end{tabular}\\\hline
  \label{table:R9}
\end{table}


%%
%%Riesgo R10
%%
\begin{table}[H]
  \centering
  \begin{tabular}{|p{15cm}|}\hline
    {\bf Riesgo}: Cese y desista por parte de LEGO.\\\hline
    {\bf Prioridad}: 1  {\bf Impacto}: 5  {\bf Probabilidad}: 0.1\\\hline
    {\bf Tipo de Riesgo}: Negocio.\\\hline
    {\bf Contexto}: Para la realización del proyecto  se utilizará LEGO Mindstorms para el diseño del robot con el cual la aplicación va a interactuar, por lo cual es importante cumplir con todos los términos y condiciones de este para así no tener problemas de tipo legales con la empresa.\\\hline
    {\bf Plan de Contingencia}: En el caso de que LEGO denegará los permisos para utilizar LEGO Mindstorms por cualquier incumplimiento de los términos y condiciones de uso, esto afectaría una parte importante del proyecto quitando gran parte de la innovación de este, y en el peor caso obligaría a cancelar este.\\\hline
    {\bf Plan de Mitigación}: En caso de no tener permisos para utilizar LEGO Mindstorms se puede:\begin{enumerate}\item Crear el robot con otro tipo de tecnologías, por ejemplo, Arduino.\end{enumerate}\\\hline
    {\bf Resolución}: Si se tiene permisos para utilizar LEGO Mindstorms para el diseño del robot se podrá seguir contando con la innovación del negocio.\\\hline
  \end{tabular}
  \label{table:R10}
\end{table}


\emph{\bf Simbología}: 
\emph{Prioridad e Impacto}: entre más cercano a 1 es menor (escala de  1 a  5). 
\emph{Probabilidad}: entre más cercano a 1 mayor. (escala de 0 a 1).
