\section{T\'ecnicas y herramientas de desarrollo.}
Para el desarrollo de este proyecto, se ha decidido utilizar distintas metodolog\'ias y herramientas, muchas de las cuales son utilizadas dia a dia por miembros del equipo. 

\subsection{Modelo de desarrollo.}

Se ha decidido usar el modelo de desarrollo RUP (Rational Unified Process) para la implementaci\'on de este proyecto. RUP es un modelo que promueve el desarrollo iterativo y organiza la elaboraci\'on de software en 4 fases (inicio, elaboraci\'on, desarrollo y cierre) las cuales consisten de una o m\'as iteraciones ejecutables de este. 

\begin{figure}[H]
  \centering
  \includegraphics[scale=0.45]{Modelo.jpg}
%  \capture{}
  \label{fig:RUP}
\end{figure}

Este proyecto se separar\'a en cuatro secciones correspondientes a cada una de las entregas de ejecutables, las cuales se profundizar\'an en las cuatro iteraciones del ciclo de desarrollo. Las secciones son:

\begin{enumerate}
\item Funcionamiento b\'asico, dise\~no de robot
\item Implementaci\'on mascota virtual
\item Interacci\'on con robot
\item Interface, fluidez de interacci\'on entre mascota virtual y robot.
\end{enumerate}

De esta forma, se llevar\'an a cabo las distintas etapas de una manera iterativa, secuencial, modularizada e incremental.

Las especificaciones de los casos de uso y requerimientos se encuentran en el archivo Anexo.xlsx

\subsection{Herramientas y t\'ecnicas de soporte para el desarrollo.}

Para el desarrollo de Viper, el equipo Phyrex ha decidido utilizar las siguientes herramientas:

\begin{table}[H]
  \centering
  \begin{tabular}{|p{5cm}|p{9cm}|}\hline
    Herramientas & Detalle \\\hline\hline
    Sistema operativo android 2.1 o superior & Plataforma oficial de la aplicaci\'on\\
    Java & Lenguaje de programaci\'on usado para la aplicaci\'on de Android\\
    C & Lenguaje de programaci\'on usado para programar robot LEGO Mindstorms NXT\\
    UML & Lenguaje de modelado de base de datos\\
    Photoshop  & Herramienta de dise\~no gr\'afico\\
    LEGO Digital Designer & Herramienta de dise\~no y armado de estructuras de LEGO\\
    Google docs & Herramienta de edici\'on colaborativa de documentos\\
    Git & Herramienta de control de versiones\\
    Android SDK tools & Herramientas de desarrollo en Android\\
    Eclipse & Entorno de desarrollo para Android\\
    SQLite & Base de datos para la aplicaci\'on\\
    LEGO Mindstorms NXT & Robot programable de LEGO\\
    LEGO Mindstorms 2.0 & Ambiente de desarrollo para Mindstorms\\
    ROBOTC for LEGO Mindstorms & Ambiente de desarrollo para Mindstorms\\
    Skype & Herramienta de comunicaci\'on entre miembros del equipo\\
    Facebook & Herramienta para comunicaci\'on de noticias del proyecto\\
    \LaTeX & Edici\'on de documentos\\
    Microsoft Project & Herramienta de creaci\'on y manejo de carta gantt\\
    StarUML & Herramienta de modelado de casos de uso\\
    Trello & Herramienta de gesti\'on de proyectos\\\hline
  \end{tabular}
\end{table}

\subsection{Personal y capacitaci\'on del equipo de desarrollo.}

Para el desarrollo del proyecto, se necesita contar con un equipo que tenga conocimientos en Java para desarrollo en Android, SQLite, ROBOTC para LEGO Mindstorms, y Photoshop. 

El equipo de desarrollo para este proyecto es la pre-empresa Phyrex, una pre-empresa formada por cinco estudiantes de ingenier\'ia civil inform\'atica de la UTFSM, los cuales se presentan a continuaci\'on:

\begin{itemize}
\item {\bf Juan Avalo}: Experiencia en los lenguajes relevantes (Java, C). 
\item {\bf Celeste Bertin}: Experiencia previa en desarrollo en Android, aprendizaje r\'apido para resolver problemas nuevos. 
\item {\bf Patricio Carrasco}: Programador con experiencia en varios lenguajes.
\item {\bf Roc\'io Fern\'andez}: H\'abil dise\~nadora gr\'afica, experiencia previa en desarrollo en android, C y librer\'ia gr\'afica AndEngine, uso avanzado de LEGO. 
\item {\bf Rodrigo Fr\'ias}: Experiencia en maquetaci\'on de textos y programaci\'on en C.
\end{itemize}

El equipo esta en constante aprendizaje de Android para sacarle el mayor provecho a esta tecnolog\'ia. El equipo esta en capacitaci\'on de ROBOTC a trav\'es de tutoriales y documentaci\'on disponible en la web. 
