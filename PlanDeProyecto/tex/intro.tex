\chapter{Introducci\'on}
Uno de los grandes problemas, hoy en d\'ia, para quienes trabajan en el \'area de la inform\'atica, es el gran desconocimiento que existe en ella. Una forma de evitar esto, es el motivar a alumnos a conocer las distintas \'areas de trabajo de los que trabajan en inform\'atica. Para ello, se han realizado distintas actividades en la Universidad T\'ecnica Federico Santa Mar\'ia (UTFSM), que hacen uso de robots, para captar la atenci\'on de su p\'ublico objetivo.

\emph{Virtual Interactive Pet Robot} (en adelante denominado \emph{VIPeR}) nace como una forma de paliar el problema de motivaci\'on y desconocimiento de la labor de quienes trabajan en el \'area inform\'atica. Se enfoca, principalmente, en responder a las necesidades de nuestro cliente, adecu\'andose a los proyectos de captaci\'on de estudiantes de Ense\~nanza Media que ya existen en la UTFSM, en su campus Santiago. Para este prop\'osito, \emph{VIPeR} se enmarca dentro de una serie de actividades que utilizan el sistema LEGO Mindstorms como forma de atraer a futuros estudiantes a la inform\'atica.

En la secci\'on 2 se mostrar\'a que ideas similares existen actualmente, indicando similitudes y diferencias con la idea propuesta. Se identificar\'an las restricciones impuestas por el cliente, adem\'as de alternativas a la soluci\'on y se encontrar\'a la que mejor se ajusta a nuestros requerimientos.

En la secci\'on 3 se definir\'a el modelo de desarrollo del proyecto, adem\'as de los elementos a utilizar junto con las competencias del equipo. En la siguiente, se indicar\'an los riesgos y el plan de contingencia o de mitigaci\'on, seg\'un corresponda.

En la secci\'on 5 se muestran los planes para el funcionamiento del proyecto, su entrega y su mantenci\'on. Finalmente en la secci\'on 6 se indica la planificaci\'on de desarrollo del proyecto
