\chapter{Implementaci\'on (entrega y operaci\'on)}
\newpage
Para poder ocupar en forma correcta \emph{VIPeR} se va a necesitar:
\begin{itemize}
\item Un smartphone con sistema operativo Android versi\'on 2.1 o superior y que tenga al menos la posibilidad de ocupar bluetooth y pantalla t\'actil.
\item Un robot LEGO Mindstorms versi\'on NXT, con al menos:
  \begin{itemize}
  \item Dos bricks. 
  \item Seis motores
  \item Lista de sensores usados por nuestro robot
  \end{itemize}
\end{itemize}

\section{Plan de operaci\'on del sistema}
Los usuarios del producto ser\'an personas portadoras de Smartphones con Sistema Operativo Android, con \'enfasis en estudiantes de ense\~nanza media. 

Cuando un usuario use \emph{VIPeR} por primera vez va a tener la posibilidad de configurar en la aplicaci\'on su mascota de acuerdo a sus gustos (nombre, tipo de mascota). La pantalla de configuraci\'on incluye la calibraci\'on de los motores y sensores de acuerdo a una configuraci\'on predeterminada de robot. Si bien est\'a dise\~nado para interactuar con un robot LEGO Mindstorms, cabe destacar que ser\'a posible hacer uso de la aplicaci\'on sin necesidad de tener un robot real. En ese caso va a tener funcionalidad limitada a las funciones disponibles en el smartphone. Los usuarios tendr\'an acceso a indicaciones sobre la instalaci\'on y uso de \emph{VIPeR}.

Phyrex, como pre-empresa responsable del desarrollo de \emph{VIPeR}, se compromete de manera \'integra con los siguientes puntos respecto al sistema:
\begin{itemize}
\item Cumplimiento total de requerimientos en plazos acordados con el cliente
\item Producto orientado al usuario, interfaz intuitiva.
\item Asistencia t\'ecnica en el uso de la aplicaci\'on, la cual puede ser entregada por medio de la website oficial o Facebook.
\item Soluci\'on pronta de errores ocurridos en la aplicaci\'on.
\end{itemize}

\newpage
\section{Plan de implementaci\'on (entrega)}
La iniciativa se basa en un plan inicial de promoci\'on informativa del producto, el cual tiene 3 fases:

\begin{enumerate}
\item Reconocimiento del producto: Dar a conocer mediante website, redes sociales (Facebook, Twitter, Google+) presentando la empresa y explicando en qu\'e consiste a grandes rasgos nuestro producto. El objetivo principal es traer a potenciales usuarios y gente interesada que permita una mayor comunicaci\'on externa y genere una imagen general de nosotros y un posicionamiento inicial enfocado en la innovaci\'on.
\item Integraci\'on de potenciales usuarios: Actualizaciones peri\'odicas sobre el producto mediante anuncios y elementos audiovisuales los cuales se situar\'an en los medios usados en fase 1. El objetivo principal es interiorizar a los interesados no s\'olo en el producto final, sino que en su mejora posterior, dando oportunidad de opini\'on a los mismos usuarios sobre posibles caracter\'isticas incorporables en \'este.
\item Consolidaci\'on: Dejar accesible la aplicaci\'on en la website oficial y luego a trav\'es de Google Play de manera gratuita. El c\'odigo necesario para que el robot interact\'ue con el smartphone va a estar disponible en una descarga aparte, la cual estar\'a presente en la p\'agina oficial.
\end{enumerate}

Una vez disponible la aplicaci\'on, la instalaci\'on se puede realizar de manera independiente por los usuarios.


\newpage
\section{Plan de mantenci\'on}
Phyrex asegura soluciones efectivas a la brevedad de posibles errores surgidos en la aplicaci\'on. Ser\'a posible reportar estos errores mediante website, redes sociales (Facebook, Twitter, Google+) en primera instancia. Tambi\'en Google Play se presenta como opci\'on una vez se realice el lanzamiento en esta plataforma.

Una vez conocidos los posibles problemas, se proceder\'a a realizar una verificaci\'on del error, y publicaci\'on de nuevas releases con la soluci\'on implementada. Se priorizar\'an requerimientos esenciales acordados con cliente y problemas de compatibilidad con dispositivos que cumplan los requisitos ya expuestos.

Phyrex no se responsabiliza por da\~nos o problemas causados por armado incorrecto de LEGO Mindstorms NXT o mal uso de la aplicaci\'on, sin embargo, se compromete a entregar servicios de asistencia y atenci\'on al cliente en armado del robot, instalaci\'on de la aplicaci\'on y uso de \'esta.

\newpage

